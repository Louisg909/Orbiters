\documentclass{article}

% Language setting
% Replace `english' with e.g. `spanish' to change the document language
\usepackage[english]{babel}

% Set page size and margins
% Replace `letterpaper' with`a4paper' for UK/EU standard size
\usepackage[letterpaper,top=2cm,bottom=2cm,left=3cm,right=3cm,marginparwidth=1.75cm]{geometry}

% Useful packages
\usepackage{amsmath}
\usepackage{graphicx}
\usepackage[colorlinks=true, allcolors=blue]{hyperref}

\title{Orbiters}
\author{Louis George}

\begin{document}
\maketitle
\text{}
\text{I start with the following code:}

\begin{verbatim}
MASS=400000
mass=3000
G=6.67408E-11
timeFrameLength=1
pos=[100,400]
POS=[200,200]
vel=[2,-1]
\end{verbatim}
\text{This is deciphered as the following:}
\begin{align*}
\text{Centre Mass} &=400000 \\
\text{Orbiter Mass} &=3000 \\
\text{Gravitational Constant} &= 6.67408\times10^{-11}\\
\text{Time Resolution} &= 1\\
\text{Orbiter Position} &=(100,400) \\
\text{Centre Position} &=(200,200) \\
\text{Orbiter Velocity} &= \begin{bmatrix} 2 \\ -1 \end{bmatrix}
\end{align*}\\\\
\text{The next block of code defines the function "Orbiter":}\\
\begin{verbatim}
    def Orbiter(pos,POS,veloc,MASS,mass):
        # finding the orbital radius
        rad=math.sqrt((POS[0]-pos[0])**2+(POS[1]-pos[1])**2)
        # getting the acceleration
        acc=(G*MASS)/abs(rad)**2
        # getting the new velocity vector
        for i in range(2):
            veloc[i]+=(acc*timeFrameLength)*((POS[i]-pos[i])/rad) #(pos[i]/rad) 
                        being to make it go towards the object
        # getting the new position
        for i in range(2):
            pos[i]+=veloc[i]*timeFrameLength
        return [pos,veloc]
\end{verbatim}
The first part of this code is working out the radius of the orbiter, using Pythagoras' Theorem:
\\$R=\sqrt{\Delta x ^2 + \Delta y ^2}$
\\\\\text{We then find the acceleration due to gravity using the following equation:}
\\$g=G\frac{M}{R^2}$
\\With $M$ being the mass at the centre, $g$ being the acceleration due to gravity and $R$ being the radius of the orbiter.
\\\\The next part (the for loop) is getting the new velocity vector for this time-frame.
\\
\begin{align*}
\vec{v}&=\vec{v}_0 + \vec{a} \Delta T\\
\begin{bmatrix}\dot{x} \\ \dot{y}\end{bmatrix} &= \begin{bmatrix}\dot{x}_{0} \\\dot{y}_0\end{bmatrix}+\begin{bmatrix}\ddot{x} \\ \ddot{y}\end{bmatrix}\Delta T
\end{align*}

I separated the acceleration into its components by multiplying it by a unit vector:
\\$\begin{bmatrix}\frac{\Delta x}{R} \\\frac{\Delta y}{R} \end{bmatrix}$
\\\\Which points towards the centre object, worked out by finding the proportion each dimension contributes to the radius.
\\\\After this we can work out the change in displacement:
\\$\vec{s}=\vec{v}\Delta T$
\\\\Finally, the first iteration the initial values are plugged into the function, and every time frame after that, the function is used with the previous values and the displacement is displayed.




\end{document}
